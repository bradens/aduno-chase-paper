\documentclass[conference]{IEEEtran}
\usepackage[utf8]{inputenc}
\usepackage{amssymb}
\usepackage{graphicx}
\usepackage{pifont}
\usepackage{multirow}
\newcommand{\cmark}{\ding{51}}%

\title{Aduno: Real-Time Collaborative Work Design In A Shared Workspace}
\author{\IEEEauthorblockN{Braden Simpson}
\IEEEauthorblockA{University of Victoria\\
Victoria, British Columbia\\
braden@uvic.ca}
\and
\IEEEauthorblockN{Eirini Kalliamvakou}
\IEEEauthorblockA{University of Victoria\\
Victoria, British Columbia\\
ikaliam@uvic.ca}
\and
\IEEEauthorblockN{Nathan Lambert}
\IEEEauthorblockA{University of Victoria\\
Victoria, British Columbia\\
nlambert@uvic.ca}}

\begin{document}
\maketitle

\begin{abstract}
In this paper we introduce Aduno, a shared workspace tool that allows distributed software teams to collaboratively establish and prioritize work items for the purposes of task management and planning during the design phase. Aduno is highly visual and real-time, offering features that seem to be lacking from other popular collaborative development tools. In line with the growing adoption of Github, Aduno also links to Github's issue tracker to easily translate work items on a whiteboard to project work items. Here, we describe the concept and design of Aduno and present its initial evaluation.
\end{abstract}

\section{Introduction \& Motivation}
\label{sec:intro}

Teams today, especially as they become distributed, need {\sc cscw} applications that support them through all phases of their collaboration by providing them with the essential context. Successful collaboration between members of a distributed team requires that people are aware of each others' actions and that they maintain this awareness so that there are as few disruptions as possible in their workflow and their coordination. It is also important that they are able to collaboratively negotiate and carry out work tasks, so tools that combine real-time editing with visual workspaces help towards commonly agreed upon task management. 

One setting that these principles take effect in is distributed software development, where teams need to coordinate for different sets of activities until they have built the end product. During the earlier stages of development, developers need to establish work items to plan and organize their work around. A comparison between some of the most popular and widely used tolls for collaborative software development shows that they mostly rely on text-based methods of establishing work items and handling workflow, not providing highly visualized features and not being particularly real-time. This might be efficient for teams that have also more informal means of communicating and negotiating work, but distributed teams mostly miss out on these indirect communication methods. We hypothesize that a highly visual, real-time environment for creating and managing work items between team members, can better support developers through the design process and can produce benefits in terms of the clarification, errors and speed of this part of the work process. 

We introduce a shared workspace application that enables visually creating and organizing work tasks in real time, targeted towards software development teams. Our solution originates from the basic concept of a shared whiteboard, incorporates communication and collaborative features that are present in the state-of-the-art tools developed for the same target group, and successfully supports group and workspace awareness as a collaborative basis in development projects. In line with the growing best practice of using Github as a collaborative platform for development management, Aduno supports synchronizing with Github's issue tracker for quickly translating work items to actionable pieces of development effort.  Aduno uses the real time collaborative aspects of whiteboards, and extends it to more distributed settings, while leveraging a persistant data source like Github.

The paper is structured as follows: We present background and work relating to building groupware applications supporting collaborative work in Section~\ref{sec:background} and explain how we used this as a foundation for Aduno's requirements. We then present the tool's conceptualization and design decisions in Section~\ref{sec:concept} and describe the initial evaluation that we carried out with a team of experts in Section~\ref{sec:evaluation}. Section~\ref{sec:results} presents the results of our evaluation. We describe our future work goals in Section~\ref{sec:future} and conclude the paper in Section~\ref{sec:conclusion}.


\section{Background \& Related work}
\label{sec:background}

In building groupware applications, effort has been placed on how to support grounding through visibility and co-presence. McCarthy et al. \cite{MCMM91} conducted experiments with teams collaboratively solving problems both with and without a shared space and found that the degree of disagreement was inversely related to the development of common ground within teams. Gutwin et al. \cite{GRG96} enhanced a shared workspace application with widgets providing activity and location information of participants and found that the add-ons provided value towards the successful completion of the tasks and the degree of awareness maintained within the team.

\begin{table*}
  \centering
  \begin{small}
    \begin{tabular}{p{9em}lcc}
      \hline
      \textbf{Category} & \textbf{Element} & \textbf{Questions} \\
      \hline
      \multirow{3}{9em}{\textbf{Who}} & Presence & Is anyone in the workspace? \\
      & Identity & Who is participating? Who is that?  \\
      & Authorship & Who is doing that? \\
      \hline
      \multirow{3}{9em}{\textbf{What}} & Action & What are they doing? \\
      & Intention & What goal is that action part of? \\ 
      & Artifact & What object are they working on? \\
      \hline
      \multirow{4}{9em}
      {\textbf{Where}} & Location & Where are they working? \\
      & Gaze & Where are they looking? \\
      & View & Where can they see? \\
      & Reach & Where can they reach? \\
      \hline
    \end{tabular}
  \end{small}
  \caption{Elements of workspace awareness(from\cite{GG02})}
  \label{tab:Gutwin}

\end{table*}


Gutwin \& Greenberg \cite{GG02} used Neisser's Perception-Action Cycle to produce a conceptual framework of the information required for workspace awareness (Table~\ref{tab:Gutwin}). The basic elements of their framework come from answering the questions of ``who, what and where''. These elements make up workspace awareness and although in a physical setting they are easy to attain, they need to be explicitly supported in groupware applications. Designers of collaborative tools need to account for workspace awareness through the features they offer. This will help a collaborating team establish and maintain common ground throughout their collaboration to ensure a successful result.
%CITE
Today we see a definite trend towards building tools that support web-based collaborative work that is real-time, enabling teams to achieve common ground faster, maintain awareness of others' actions more effectively, and coordinate more smoothly. As only one successful example, Google Docs is a web-based text, spreadsheet, presentation and form editor whose data storage is provided by Google. Google Docs~\cite{SIRM07} is a truly collaborative tool for document editing, allowing sharing and editing by multiple users at the same time. The design features offered by Google Docs are awareness solutions that allow collaborators to have real-time information on the activity and location of other users in the form of remote cursors, participant lists and chat, revision history, etc.

Distributed software development teams face the same challenges of distributed collaboration in terms of grounding and maintaining workspace awareness. Especially during the initial stages of a software development project, effort is required to define and prioritize work items as well as assign or negotiate who is responsible for what. As distributed teams aim towards leaner organizational structures and modularized work items to overcome some of the challenges of distribution \cite{Herbsleb07, HG99, Parnas72}, tools that enable teams to define work items collaboratively, as a starting point for their development work, are essential. The underlying rationale here is that the modularity of the product design will decide the modularity of the work tasks and if this process can be as transparent and as commonly agreed upon as possible in the project, it will set the stage for smoother coordination down the road. This rationale is also in line with Conway's Law \cite{Conway68} which is as relevant to distributed software teams today as ever.

Based on this background knowledge, the design of the tool was aimed to support distributed teams during the design phase when work item creation and management are crucial. Our goal is to implement features that visually provide workspace awareness to team members, and embed them on a web-based infrastructure to allow for real-time collaboration. Our tool also promotes a more flat and flexible team structure, allowing team members to negotiate and establish roles through self-organization by working on specific work items, or outside the tool.

Requirements for building Aduno came from reflecting on the background knowledge on {\sc cscw} applications and the features that successful groupware should provide. 


\subsection{{\sc cscw} application success and failure}
Grudin \cite{Grudin88} discussed the main reasons behind the failure of {\sc cscw} applications. Through reviewing cases of {\sc cscw} application areas that have fallen short of expectations, he concluded that there are three factors that can lead {\sc cscw} applications to not be adopted. Below, we summarize his insights and highlight how these created the initial requirements for Aduno.

\textbf{Disparity between effort and benefit.} Using automatic scheduling applications as an example, Grudin identified that the application's use was not mutually beneficial to the groups involved, and this negatively affected adoption. An automatic scheduler for meetings was beneficial to the time-conscious manager that initiated meetings, but only required extra effort on behalf of group members that were positioned lower in the organizational hierarchy. The latter wouldn't have ordinarily kept up with automatic schedulers, but were now required to do so to provide benefit to the regular user group of this application. Grudin's view was that the underlying cause of the problem was the erroneous analogy of the single-user application model used for groupware.

This kind of mismatch created a fundamental initial requirement for Aduno. As a tool intended to support groups, it is designed to be used by the entire group and \textit{provide mutual benefit}. Identifying, creating, and linking tasks in a real-time manner that makes the whole group aware of the process while leaving room for task negotiation since it is done collaboratively, brings benefit to all users involved in the software development project, from the project manager to the developer.

\textbf{Difficulty of {\sc cscw} application evaluation.} Aduno is by design aimed towards group use, and disregards group hierarchy or roles inside the tool.  Dourish et al. found that having dynamic role creation in the collaborative application was a positive factor\cite{Dourish:1992:ACS:143457.143468}, and this was an aspect that was incorporated into Aduno. Grudin's view of evaluating {\sc cscw} applications was that it is more complex because it involves multiple users and different groups. He proposed drawing on social science techniques to help with identifying group dynamics and trying to incorporate this into the evaluation. Aduno, having disposed of hierarchical roles inside the tool, creates \textit{less demands for using specialized techniques}. Aduno's evaluation was done through en expert study where its usability and fit for the intended users was verified.

\subsection{Framework describing workspace awareness}
The next background element that impacted Aduno's functional requirements is Gutwin's \& Greenberg's \cite{GG02} framework of information to support workspace awareness for real-time groupware. The framework is intended to aid the design of real-time groupware and makes use of different theories and observational studies. The framework is especially relevant to Aduno's design as it targets real-time distributed groupware, shared workspaces and both generational and execution tasks. In terms of group size, the framework mainly discusses small groups. For the current design state of Aduno this fits the tool's purpose, although one of the future work items for Aduno is to investigate how it scales in order to be used by larger groups.

Referring back to Table~\ref{tab:Gutwin}, awareness relating to the ``Who'' category sets the aim for groupware to provide information of others present in the workspace and who they are, while authorship information maps an action to the person performing it. Regarding the ``What'' category, a tool must successfully communicate information about what other users are working on, either in detail or in general. Aduno's features, as discussed in Section~\ref{sec:concept}, cover these two categories, fulfilling the requirements for real-time groupware supporting awareness. Information relating to the ``Where'' category, is intended for large workspaces and better fitted to construction tasks, such as the one discussed in \cite{GRG96}. To this end, they are not explicitly part of Aduno's design. However, location and gaze information can be inferred by the work items a user is moving around the workspace.

In the following section we describe how we conceptualized Aduno after our initial requirements.

\section{Conceptualizing \& developing Aduno}
\label{sec:concept}

\begin{figure*}[t]
\includegraphics[width=\textwidth]{aduno-screenshot}
\caption{Aduno Screenshot}
\label{fig:adunoscreenshot}
\end{figure*}

\begin{table}[h]
\begin{center}
\begin{tabular}{@{\hspace{.2cm}}ccc@{\hspace{.2cm}}c@{\hspace{.2cm}}c@{\hspace{.2cm}}c@{\hspace{.2cm}}c@{\hspace{.2cm}}}
\hline
Similar Services&  Real-time&   Control of&  Tags&    Visual layout&      Chat&\\
 & & WorkItems& & & &\\
\hline
GitHub Issues   &	-&	        \cmark&	                \cmark& -&                  -\\ 
Google Code     &   -&          \cmark&                 -&      -&                  -\\
Jazz            &   -&          \cmark&                 \cmark& -&             -    \\
MindMeister & \cmark& -& -& \cmark& - \\
Trello & \cmark& -& \cmark& -& \cmark\\
Aduno           &   \cmark&     \cmark&                 \cmark& \cmark&             \cmark\\
\hline
\end{tabular}
\end{center}
\caption{Comparing Collaborative Tools\label{tab:services}}
\label{tab:otherservices}
\end{table}

Other collaborative development environments (CDEs) have some of their features mapped in Table~\ref{tab:otherservices}.  They provide an almost exclusively asynchronous approach to developing articulative work, and this is one of the motivators for Aduno.  Aduno poses to interface with these other CDEs and create functionality to create semi-synchronous collaborative experience.  GitHub was chosen as the preferred backend because of it's robust and open API, in addition to it hosting almost three million open-source repositories.

\subsection{Purpose}
Aduno was constructed around the requirements described in Section~\ref{sec:background}.  From this model, the requirements were gathered and formulated into implementation.  The main aspects of collaborative applications that were the basis of the architectural design decisions of Aduno are \emph{speed}, \emph{synchronicity}, and \emph{simplicity}.  With these components in mind, the development team built Aduno to be able to interface with the GitHub API, or other RESTful services, to provide a faster, more effective way to collaborate on articulative tasks. 

\subsection{Feature Description}
Figure~\ref{fig:adunoscreenshot} shows a typical example of Aduno's interface, with a few emphasized features: {\bf (1)} The top navigation bar with a menu to select the repository to work on, and an indicator showing other online collaborators. {\bf (2)} Tabbed workflow panel that lets users switch between label list, milestone list, and chat area. {\bf (3)} Label filters that let users show only specific work items based on a specific label {\bf (4)} Work items that can be edited in real-time by several users. They can be linked to other work items, as shown in the screenshot. This allows for a visual representation of links, which is not offered by other services.

Although it is typical for project management systems in services such as GitHub to allow comments on each issue (herein referred to as the more generic term \textit{work item}), we have included a chat area in Aduno. The comments on work items in Github are used for asynchronous communication, and this was not sufficient for our tool. A chat system provides users with a secondary communication channel that allows them to engage in a semi-synchronous discussion about work items that would not typically be appropriate for asynchronous work item comments. Since comments in Github must also be attached to only one work item, having this chat area helps teams develop broader goals and facilitates common ground through more in-depth conversation.

As noted on Figure~\ref{fig:adunoscreenshot}, one of the paramount features of Aduno is the ability to organize work items on the workboard (the "canvas" on which the work items are displayed). In other services, work items are displayed in a list without any notion of priority. Although they allow for links between work items, users must follow a link to view the other work items. Having a visual representation of the links allows users to quickly glance at the related work items. Allowing users to move the work items around on the work board gives them several opportunities, including being able to arrange them in whatever context that seems appropriate at the time. Work items may be ordered chronologically, in order of importance, or in whatever order the team wishes. Work items may not only be linked, but they may also be grouped together to imply a connection. Links are part of the Aduno API, so they can be passed on to any associated services that support this feature, such as GitHub.

When an edit is performed on a work item in Aduno, the "Sync" indicator at the top right of the item turns from green to orange to signal that there is an unsynchronized change. To synchronize with GitHub, the user may click on the Sync All button at the top of the page, or they may click on a work item's Sync button to synchronize it individually. This gives the team more control over what work items get synced with the associated service.

The elements of workspace awareness outlined in Table~\ref{tab:Gutwin} should be completely fulfilled in an effective collaborative application, although some of them are not applicable to this tool. As previously mentioned, awareness of gaze, view, and reach would add much complexity to the tool without providing much benefit. However, the remaining elements play an important role in user awareness with Aduno.
  
{\bf Presence} – An indicator in the navigation bar at the top of the screen signals the presence of other users.  Also, whenever users are interacting with controls in the workspace, they have their presence made in the form of their unique color and username badge shown.

\begin{figure}[htb]
\centering
\includegraphics[scale=0.5]{aduno01}
\caption{Online user indicator}
\label{fig:otherusers}
\end{figure}

{\bf Identity} – Clicking on the indicator shows a dropdown listing of all the current online users. A color unique to that user is displayed next to their name.  This is a globally used identity that is persistent to all items related to the user. 

{\bf Authorship} – When a user is editing a work item, their name is displayed at the bottom of it to indicate that they are currently working on it.

\begin{figure}[htb]
\centering
\includegraphics[scale=0.5]{aduno02}
\caption{Work Item Being Edited}
\label{fig:workitem}
\end{figure}

{\bf Action} – Editing actions result in real-time data propagation, immediately showing changes across all clients that are connected to the same repository.  

{\bf Intention} – The intention of most work item edits would likely be apparent as the edits are being done. Broader goals may be discussed in the chat window. A markdown language is used on the chat to provide contextual awareness to the users.  For example someone can reference a work item by using it's ID in the chat window.
 
{\bf Artifact} – As shown in Figure~\ref{fig:workitem}, usernames are displayed at the bottom of it to indicate that they are currently working on an item.

{\bf Location} – The location of the user’s work is shown by the username attached to a work item when they are doing edits. If the user moves a work item around, their location becomes apparent.

{\bf Ownership} - Storing a history of modifications to work items is a way to add awareness for users that are not synchronously working, and allows for reflection on the creation process.

\subsection{Architecture}
With \emph{speed} being one of the most important aspects of the design, the architecture chosen for the server-side technology is crucial.  Node.js\footnote{A platform for building fast, scalable network applications, http://nodejs.org}, a platform of server-side JavaScript built on Google Chrome's V8 JavaScript runtime, was chosen as our platform.  Node has been quickly adopted as a powerful platform for web applications such as Cloud9\footnote{A collaborative web IDE built in Node.js}, LinkedIn, Windows Azure and more.  Tilkov \& Vinoski~\cite{TV10} have shown Node as a viable choice for non-blocking web applications.  Using Node allows Aduno to be deployed easily to the Amazon Elastic Cloud compute instances as a cloud service, which offers reliability and scalability.

\emph{Synchronicity} is a crucial component of Aduno and was a great influence over the frameworks chosen for the tool.  The Meteor Framework\footnote{An open-source web application framework focused on real-time updates} was chosen to support asynchronous data among users, since Meteor pushes changes out to clients dynamically, updating their screens without refreshing, and unobtrusively keeping users synchronized.  

\emph{Simplicity} is a key point in any application that needs to attract its new users, and Aduno had this principle as a top priority.  The web toolkit used to support this is Twitter Bootstrap, a very common open-source user interface toolkit, with approximately 100 developers.

\section{Evaluation}
\label{sec:evaluation}

To evaluate Aduno's usability and fit for the intended use, we carried out an expert judgement study involving eight users, forming four pairs. The experts had experience in distributed software development ranging from one to ten years and have participated in multiple collaborative projects.

\subsection{Setup}
The experts were invited to participate in the evaluation session and, after indicating their availability, were randomly assigned to pairs. In each evaluation session, the experts were first given a short demonstration of the tool's features and the shared workspace layout. Following that, participants were given a short description and high-level requirements for a fictional website they would have to build in collaboration with their partner. They were asked to use the tool to collaboratively define and link work items. The experts were not allowed to communicate outside the tool, as a way to account for them being distributed. For the final part of the evaluation session, experts were given a short questionnaire to fill out, evaluating the tool's fulfillment of its purpose as well as its potential impact on a project.

\subsection{Questionnaire}
The questionnaire's design was guided by insights from the literature regarding collaborative behaviours that should be successfully supported by {\sc cscw} applications. Clark \cite{Clark96} defined the following eight behaviors for collaboratively carrying out joint actions:

\begin{itemize}
\item \textbf{Connection} -- locating with whom to collaborate and how to contact them
\item \textbf{Transmission} -- sending a message
\item \textbf{Notification} -- alerting the intended party of an incoming transmission
\item \textbf{Identification} -- designating the sender, receiver, and subject of a transmission
\item \textbf{Common Ground Preservation} -- establishing and maintaining a shared context and meanings in transmissions
\item \textbf{Confirmation} -- notifying the sender of a transmission that it has been received
\item \textbf{Synchronization} -- orchestrating actions to facilitate joint action
\item \textbf{Election} -- group process of selecting among alternatives
\end{itemize}

Thompson \cite{Thompson67} defined three types of coordination. \textbf{Standardized} coordination includes a full set of established rules through which people have to coordinate their activities (an example being traffic rules). \textbf{Planned} coordination requires team members to plan their coordination process based on the task at hand. Planned coordination is different for software development tasks versus writing a movie script, for example. When there are emergent activities as a response to the demands of the collaborative task, the coordination is said to be of the \textbf{mutual adjustment} type. Software development fits under this type of coordination as requirements and the development context can prove to be quite dynamic and/or unpredictable.

A combination of Thompson's \cite{Thompson67} definition of different types of coordination and Clark's \cite{Clark96} collaborative behaviors resulted in a checklist of behaviors to be supported for each coordination environment (Table ~\ref{tab:collabchecklist}).

\begin{table}[h]
\begin{center}
\begin{tabular}{@{\hspace{.2cm}}lccc@{\hspace{.2cm}}c@{\hspace{.2cm}}c@{\hspace{.2cm}}c@{\hspace{.2cm}}}
\hline
Collaborative Behaviours&  Mutual Adj.&   Planned&  Standardized&\\
\hline
Connection & \cmark& \cmark& -&\\
Notification & \cmark& -& -&\\
Identification & \cmark& \cmark& \cmark&\\
Common Ground Preservation & \cmark& -& -&\\
Transmission & \cmark& \cmark& \cmark&\\
Confirmation & \cmark& -& -&\\
Synchronization & \cmark& -& -&\\
Election & \cmark& -& -&\\
\hline
\end{tabular}
\end{center}
\caption{Collaborative behaviours and coordination \label{tab:collabchecklist}}
\end{table}

Mutual adjustment coordination tasks include all collaboration behaviors. As a result, tools such as Aduno, aimed towards supporting this kind of tasks, must include features that cover the respective behaviors. To account for this during the evaluation, the questionnaire was designed to check how well these behaviours were supported. The questionnaire used can be found at [where?]

The questionnaire consisted of four sections and a total of twenty five questions. In the first section the experts were asked to provide their development experience in both number of years and number of projects they have participated in. They were also asked to indicate the tools or platforms that they are currently using for task creation and management. There was also a yes/no question of whether the experts have so far experienced difficulties with managing work tasks in previous projects. 

In the second section of the questionnaire, the experts were asked to rate Aduno's usefulness for visualizing work items, brainstorming, and translating ideas to actionable items. A 5-point Likert-like scale was used for rating, ranging from ``Not useful" to ``Very useful". Experts were also asked to rate Aduno's ease to use, learn, and integrate with their current workflow, using the same type of scale ranging from ``Very difficult" to ``Very easy".

The third section of the questionnaire asked the experts to rate the importance of Clark's~\cite{Clark96} collaborative behaviours and their satisfaction by Aduno's support for them. The experts were provided with the names and definitions of the eight behaviours, that were slightly modified to reflect the application domain, i.e. software development. Importance and satisfaction were rated on a 5-point Likert scale ranging from ``Trivial" to ``Critical" for importance, and ``Not satisfied" to ``Fully satisfied" for satisfaction. 

For the fourth and final section of the questionnaire, experts were asked to describe their expectations for Aduno. Supposing that they would adopt Aduno for their projects, experts were asked about the impact they would expect on team productivity as well as challenges expected for their workflow. Finally, experts were asked to indicate their intent to adopt Aduno and suggest future improvements and/or development areas for Aduno. In the next section we present and discuss the evaluation results.

\section{Results}
\label{sec:results}

\begin{table*}[t]
\begin{center}
\begin{tabular}{@{\hspace{.2cm}}llll@{\hspace{.2cm}}c@{\hspace{.2cm}}c@{\hspace{.2cm}}c@{\hspace{.2cm}}}
\hline
Collaborative Behaviours&  Mean&   Median&  St. Dev.&\\
\hline
Visualizing work items& 4.125& 4& 0.83\\
Brainstorming& 3.375& 4& 1.19\\
Translating ideas to actionable items& 4& 4& 1.07\\
Easy to use& 4& 4& 0.75\\
Easy to learn& 4.75& 5& 0.46\\
Easy to integrate with current workflow& 3.875& 4& 0.83\\
\hline
Importance of Awareness& 4.375& 4.5& 0.77\\
Satisfaction with Awareness& 3.5& 3& 0.92\\
\hline
Importance of Connection& 3.875& 4& 1.12\\
Satisfaction with Connection& 3.25& 3& 1.03\\
\hline
Importance of Transmission& 3.625& 4& 1.5\\
Satisfaction with Transmission& 3.625& 3.5& 0.74\\
\hline
Importance of Notification& 3.25& 3& 1.03\\
Satisfaction with Notification& 2.5& 2.5& 1.19\\
\hline
Importance of Ownership& 3.375& 3.5& 1.4\\
Satisfaction with Ownership& 3& 3.5& 1.19\\
\hline
Importance of Common Ground& 4.25& 4& 0.7\\
Satisfaction with Common Ground& 3.5& 3.5& 0.92\\
\hline
Importance of Responsiveness& 4.5& 5& 0.75\\
Satisfaction with Responsiveness& 2.75& 2& 1.39\\
\hline
Importance of Shared Information Space& 4.375& 5& 0.91\\
Satisfaction with Shared Information Space& 3.875& 4& 1.12\\
\hline
\end{tabular}
\end{center}
\caption{Collaborative behaviours and coordination}
\label{tab:surveyresults}
\end{table*}

Overall, Aduno impressed the experts and received positive feedback. The results of the expert study, and their related descriptive statistics, are shown in Table~\ref{tab:surveyresults}. 
Aduno was considered a useful tool, receiving its highest score for visualizing work items. The experts also appreciated how easy Aduno is to learn (4.75 mean), use (4 mean), and integrate into their current workflow (3.875 mean).

As mentioned in Section~\ref{sec:evaluation}, the experts were asked to rate the importance they placed on Clark's \cite{Clark96} collaborative behaviors, and their satisfaction with Aduno's support for them. Experts assigned their highest ranking to the \textit{responsiveness} of the application (4.5 mean), relating back to the requirement for real-time environments. \textit{Maintaining awareness} and \textit{sharing an information space} were also ranked as very important elements for collaborative software development tasks with a mean of 4.375. Given the frameworks we reviewed in Section~\ref{sec:intro}, this is not a surprising demand for a {\sc cscw} tool. At the same time Aduno also scored high in terms of the experts' satisfactions with supporting the latter two elements, as shown by the respective means. The lower score of 2.74 obtained by Aduno with regard to \textit{responsiveness} is only relevant in this early stage of development and mostly relates to the choice of architecture and infrastructure rather than the tool's functionality. We expect that in the next version of the tool the responsiveness issues will be resolved. 

%not sure this is ok, but we should provide some kind of justification for the lower score 

In addition to the closed-ended questions, the experts were asked to evaluate Aduno's potential impact on a team's productivity in a real project. Six out of eight respondents indicated that they expect a positive impact, especially in terms of offering transparency during the design phase. The experts also commented on the ability to have commonly agreed on tasks, and how this positively affects the structure and organization of the project. The remaining two experts recognized Aduno's potential but would like it to cater to larger groups and be more explicit with milestones to have a strong positive effect.

Regarding challenges, the experts were only concerned with the scalability of Aduno and whether it will be able to support large groups and a large number of work items. Participants were also asked to provide suggestions and improvements for future development on Aduno, which will be discussed in Section~\ref{sec:future}.

\section{Future Work}
\label{sec:future}

Future work for Aduno will be carried out on two levels: extending the tool's features based on the experts' feedback, and carrying out a quantitative evaluation. 
\subsection{Development}
During this study Aduno has been evaluated using an expert user evaluation, which has provided a large amount of feedback for the development team and has highlighted many new features.  Apart from small user interface improvement suggestions, the main points of interest were as follows:

\begin{itemize}
    \item{More robust support for large number of work items}
    \item{Stronger chat Features} 
    \item{More robust API mapping to GitHub}
\end{itemize}

Although this input from the expert evaluators is extremely useful, these types of feature requests were anticipated due to the prototypical nature of the tool.  The future work on Aduno will include a significant amount of development in these areas, as well as potentially interfacing the backend with another service such as IBM Jazz\footnote{Project lifecycle management tools by IBM -- https://jazz.net/story/about}, or Atlassian Jira\footnote{Production grade project tracking system -- http://atlassian.com/software/jira/overview}. Due to the modular design of the Aduno API, the developers can easily plug in new services.  

Larger teams is another focus of the future development.  As shown in Section~\ref{sec:results}, the results of the evaluation were positive, but the experts wanted a lot more support for larger projects -- or more work items to be seen at once.  This is a crucial feature for Aduno that has to, and will be addressed.  The implementation could include a zoomable work board, which could act as a map that can be scrolled and interacted with.  

\subsection{Evaluation}
We plan to do a more extensive evaluation of Aduno in terms of user engagement, and also perform a quantitative analysis to investigate how its use affects team productivity. One idea to attract more users for our evaluation is to import data from their projects on GitHub into Aduno and then contact them with a demo of the tool with their own data. This approach will motivate the team by giving them a real data snapshot of their issue tracker already running in Aduno and can instantly show them the value of the tool.  

With regard to the quantitative study, we plan to assess the impact of Aduno's use on a team's speed and the need for clarification. We essentially want to answer the research question:

\textit{Can we effectively use real-time collaborative tools for task management in distributed software development?} 

Since Aduno helps the team collaboratively define and link work items, especially during the initial stages of the project, we expect less time spent for clarifications in later stages of development and an increase in development speed associated with that. Our research question breaks into the following two questions:

\textit{Does the use of real-time collaborative tools for task management in the design phase speed up the following stages of development?}

\textit{Does the use of real-time collaborative tools for task management in the design phase lead to fewer clarification requests in the following stages of development?}

Aduno aims towards consistently supporting the development team through their design work and, as a result, minimize the timespan for clarification activities and increase their productivity. We plan to design a field study around these two research questions, using geographically distributed teams. 

\section{Conclusion}
\label{sec:conclusion}
This paper introduced Aduno, a real-time collaborative tool for defining and managing work items, especially during the design phase, in distributed software development teams. The aim behind the tool is to add a strongly visual support for task management and make it more real-time, as these are features that seem to be missing from popular collaborative development tools. The tool's features fulfill the major requirements for collaborative applications as they are known from the {\sc cscw} and groupware literature, and were evaluated by experts. The evaluation showed that experts appreciate the visual and real-time features that Aduno adds to the functionality also supported by other tools, and expect that the use of these features will bring benefits to a software development project. Future plans for the tool include expanding the feature list as well as doing a quantitative evaluation to measure its impact on team productivity.

\bibliographystyle{IEEEtran} 
\bibliography{aduno}

\end{document}
